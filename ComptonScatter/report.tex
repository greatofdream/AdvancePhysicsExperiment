\documentclass[10pt]{ctexart}
\usepackage{graphicx, amsmath}
\usepackage{siunitx}
\usepackage{caption}
\usepackage{url}
\usepackage[version=4]{mhchem}
\title{康普顿散射实验}
\author{张爱强\\指导教师: 张钊}
\date{}
% set the section title format
\ctexset{
    section/format  += \raggedright
}
% The following parameters seem to provide a reasonable page setup.
\topmargin 0.0cm
\oddsidemargin 0.2cm
\textwidth 16cm 
\textheight 21cm
\footskip 1.0cm
% set the abstract format
\newenvironment{sciabstract}{%
\begin{quote} \textbf{摘要: }}
{\end{quote}}
% set the bibliography format
\bibliographystyle{elsarticle-num}


\begin{document}
\maketitle
\begin{sciabstract}
    本实验使用塑料闪烁体和\ce{NaI}闪烁体符合测量康普顿微分散射截面和散射角的关系。利用\ce{^{57}Co}放射源衰变后准直的$\gamma$入射进入塑料闪烁体,
    发生康普顿效应,利用\ce{NaI}闪烁体测量散射光子能谱。调整不同的角度,得到散射角和微分散射截面之间的关系。
    \par\textbf{关键词: } 康普顿散射; 符合方法.
\end{sciabstract}
\section{引言}
康普顿散射验证光的粒子性,该研究在1927年获Nobel Prize。

$\gamma$光子能量比外层电子的结合能大的多,可以把外层电子看作自由电子。利用动量能量守恒可以获得散射$\gamma$光子和散射角的关系。散射$\gamma$光子能量
\[E_{\gamma'}=\frac{E_\gamma}{1+\frac{E_\gamma(1-\cos{\theta})}{m_0c^2}}\]
用Dirac方程可以得到计算康普顿散射的微分截面的Klein-Nishna公式:
\[\frac{d\sigma(\theta)}{d\Omega}=\frac{r_0^2}{2}(\frac{E_{\gamma'}}{E_\gamma})^2(\frac{E_{\gamma'}}{E_{\gamma}}+\frac{E_{\gamma}}{E_{\gamma'}}+\sin^2{\theta})\]

\section{实验}
\subsection{实验仪器}
\begin{description}
    \item[放射源铅室] 内部有放射源,射线从准直器出射,保证射线方向。后侧有一个旋钮,旋转180\si{degree}控制准直器出口的打开与关闭。 
    \item[\ce{NaI(Tl)}闪烁体] 测量散射光子的能谱
    \item[符合器件] 对两路信号进行符合。
\end{description}

\subsection{实验内容}
\begin{enumerate}
    \item 在散射角$\theta=60^\circ$,用\ce{^{137}Cs},\ce{^{60}Co}对探测器\ce{NaI}进行能量刻度。
    \item 在不同散射角分别用符合法测量\ce{^{137}Cs}的$\gamma$能谱,定出全能峰总计数
    \item 求不同$\theta$的散射$\gamma$能量以及散射微分截面的相对值。
\end{enumerate}
\section{实验结果及讨论}
\subsection{能量刻度}

用多道直接测量\ce{NaI(Tl)}闪烁体探测器输出的能谱为图~\ref{fig:EnergyPeakFix}。\ce{^{137}Cs},\ce{^{60}Co}对应于三个特征峰,分别为
\SI{0.662}{MeV},\SI{1.1732}{MeV},\SI{1.3325}{MeV}。从道址90到300之间寻峰,可以排除掉前面的低能本底和后面大部分无用的能量区间,其次
寻找峰位满足大于左右两侧4个道址,而且距左边10个道址处的幅度要下降为峰值的2/3,筛选出如图~\ref{fig:EnergyPeakFix}中黑色三角形指示的三个位置。
将这三个道址和对应的能量使用线性函数拟合。
\begin{figure}[htbp]
    \centering
    \begin{minipage}{0.45\textwidth}
        \centering
        \includegraphics[width=\textwidth]{data/scaleEnergy.png}
        \caption{寻峰}
        \label{fig:EnergyPeakFix}
    \end{minipage}
    \qquad
    \begin{minipage}{0.45\textwidth}
        \centering
        \includegraphics[width=\textwidth]{data/scaleEnergyFit.png}
        \caption{能量刻度}
        \label{fig:EnergyFix}
    \end{minipage}
\end{figure}
图~\ref{fig:EnergyFix}三个特殊的峰的位置对应的道址分别为$18,76,$,利用线性函数进行刻度,拟合得到道址和能量之间的关系。
\[E(MeV)=(0.00697\pm 0.00011)x+(0\pm 0.58)\]
\subsection{探测效率和峰总比与能量的关系}
\label{sec:scaleResult}
使用多项式进行插值,为了避免过拟合,设置多项式最大阶数为4,拟合对应的插值函数。
\begin{figure}[htbp]
    \centering
    \begin{minipage}{0.45\textwidth}
        \centering
        \includegraphics[width=\textwidth]{data/epsilon.png}
        \caption{探测效率和能量}
        \label{fig:epsilon}
    \end{minipage}
    \qquad
    \begin{minipage}{0.45\textwidth}
        \centering
        \includegraphics[width=\textwidth]{data/R.png}
        \caption{峰总比和能量}
        \label{fig:R}
    \end{minipage}
\end{figure}
图~\ref{fig:epsion}和图~\ref{fig:R}利用线性函数进行刻度,拟合得到探测效率和峰总比与能量之间的关系。
\[\epsilon=(10.13\pm0.32)+(17.58\pm 4.47)x+(-117.5\pm19.1)x^2+(193.2\pm32.1)x^3+(-101.8\pm18.2)x^4\]
\[R=(1.023\pm0.013)+(0.52\pm 0.21)x+(-8.935\pm0.099)x^2+(15.65\pm0.15)x^3+(-8.29\pm0.14)x^4\]
\subsection{康普顿散射光子能量拟合}
通过寻峰和能量刻度结果可以获得不同$\theta$下对应的道址和能量如图~\ref{fig:theta20}等和表\ref{tab:gammaEnergy}.

\begin{figure}
    \centering
    \includegraphics[width=0.6\textwidth]{data/20.png}
    \caption{$\theta=20^\circ$}
    \label{fig:theta20}
\end{figure}
\begin{figure}
    \centering
    \includegraphics[width=0.6\textwidth]{data/40.png}
    \caption{$\theta=40^\circ$}
    \label{fig:theta40}
\end{figure}
\begin{figure}
    \centering
    \includegraphics[width=0.6\textwidth]{data/60.png}
    \caption{$\theta=60^\circ$}
    \label{fig:theta60}
\end{figure}
\begin{figure}
    \centering
    \includegraphics[width=0.6\textwidth]{data/80.png}
    \caption{$\theta=80^\circ$}
    \label{fig:theta80}
\end{figure}
\begin{figure}
    \centering
    \includegraphics[width=0.6\textwidth]{data/100.png}
    \caption{$\theta=100^\circ$}
    \label{fig:theta100}
\end{figure}
\begin{figure}
    \centering
    \includegraphics[width=0.6\textwidth]{data/120.png}
    \caption{$\theta=120^\circ$}
    \label{fig:theta120}
\end{figure}
因为探测器会有一个能量分辨率,所以全能峰对应于一个高斯分布;另外由于本底的存在,全能峰对应部分有一个本底的叠加。利用高斯和二次函数本底\cite{nuclear}叠加拟合得到
拟合结果分别如图~\ref{fig:theta40}等,对应的全能峰计数标记在表\ref{tab:gammaEnergy}

\begin{table}
    \begin{tabular}{|c|c|c|c|}
        \textbf{$\theta$} & \textbf{道址}& \textbf{全能峰计数}& \textbf{$E_{\gamma}$}\\
        \hline
                 20   &87.455606 & 6355.6201 &0.610\\
                40 & 71.267048 & 4483.3065 &0.497\\
                60 & 54.747531 & 3653.1918 &0.382\\
                80 & 42.312009 & 3114.9581 &0.295\\
                100 & 33.081857 & 3053.1071 &0.231\\
                120 & 26.404983 & 3282.8744&0.184\\
    \end{tabular}
    \centering
    \caption{角度与道址,全能峰计数,能量}
    \label{tab:gammaEnergy}
\end{table}
理论给出的散射$\gamma$光子能量与散射角的关系为公式~\ref{equ:scatterEnergy}
\begin{equation}
    E_{\gamma'}=\frac{E_\gamma}{1+\frac{E_\gamma(1-\cos{\theta})}{m_0c^2}}
    \label{equ:scatterEnergy}
\end{equation}
将对应的能量和角度进行拟合,拟合过程中保留$E_\gamma$为未知量,可以获得如图~\ref{fig:thetaEnergy}中实线的拟合结果,其中$E_\gamma=0.636MeV$,与实际上
的\SI{0.662}{MeV}有一些差距。图中虚线为理论公式~\ref{equ:scatterEnergy}给出的结果,利用理论上的分布和测量的结果做卡方检验,得到卡方值为$\chi^2=0.00387$,在自由度为4的
卡方分布下对应的p值为$p=1-1.87E-6$,所以可以认为理论上的分布与实验是吻合的。
\begin{figure}[htbp]
    \centering
    \includegraphics[width=0.7\textwidth]{data/scatterPhoton.png}
    \caption{散射光子和角度之间的关系}
    \label{fig:thetaEnergy}
\end{figure}
\subsection{散射微分截面}
理论给出的散射微分截面与散射角的关系为公式~\ref{equ:scatterSection}
\begin{align}
    \frac{d\sigma(\theta)}{d\Omega}&=\frac{r_0^2}{2}(\frac{E_{\gamma'}}{E_\gamma})^2(\frac{E_{\gamma'}}{E_{\gamma}}+\frac{E_{\gamma}}{E_{\gamma'}}+\sin^2{\theta})\\
    &=\frac{r_0^2}{2}(\frac{1}{1+\frac{E_\gamma(1-\cos{\theta})}{m_0c^2}})^2(\frac{1}{1+\frac{E_\gamma(1-\cos{\theta})}{m_0c^2}}+1+\frac{E_\gamma(1-\cos{\theta})}{m_0c^2}+\sin^2{\theta})\\
    \label{equ:scatterSection}
\end{align}
考虑到实验测定时探测器对散射样品有一定的立体角,所以接收到的光子数和立体角相关,通过取相对微分散射截面可以消除这个参数影响。
测定的散射光子的全能峰计数和实际接收到的光子数有探测效率和峰总比的因子,和可以用部分\ref{sec:scaleResult}中的刻度结果进行插值。

那么选择一个参考点后,对应的相对微分散射截面公式为
\begin{align}
    \frac{d\sigma(\theta)}{d\Omega}_{relative}=\frac{(\frac{1}{1+\frac{E_\gamma(1-\cos{\theta})}{m_0c^2}})^2(\frac{1}{1+\frac{E_\gamma(1-\cos{\theta})}{m_0c^2}}+1+\frac{E_\gamma(1-\cos{\theta})}{m_0c^2}+\sin^2{\theta})}{(\frac{1}{1+\frac{E_\gamma(1-\cos{\theta_r})}{m_0c^2}})^2(\frac{1}{1+\frac{E_\gamma(1-\cos{\theta_r})}{m_0c^2}}+1+\frac{E_\gamma(1-\cos{\theta_r})}{m_0c^2}+\sin^2{\theta_r})}
    \label{equ:scatterSectionRelative}
\end{align}
而实验的数据处理得到相对微分散射截面公式为
\begin{align}
    \frac{d\sigma(\theta)}{d\Omega}_{relative}=\frac{N_p(\theta)R(\theta_r)\epsilon(\theta_r)}{N_p(\theta_0)R(\theta)\epsilon(\theta)}
    \label{equ:scatterSectionRelativeExp}
\end{align}
选择$60^\circ$作为参考点,画出相对散射微分截面与角度之间的关系,如图~\ref{fig:scatterSection}中散点是实验数据给出的结果,虚线是理论给出的结果
\begin{figure}[htbp]
    \centering
    \includegraphics[width=\textwidth]{data/scatterPBarn.png}
    \caption{相对微分散射截面}
    \label{fig:fit}
\end{figure}
可以看出在角度较小的时候,理论给出的结果和实验结果差异较大,说明实验测量的光子数目明显大于理论预期,可能和几何因素相关,因为探测器有一定尺寸,
在角度较小时,一些光子可以直接打到探测器上。

利用理论上的分布和测量的结果做卡方检验,得到卡方值为$\chi^2=2。1941$,在自由度为4的
卡方分布下对应的p值为$p=0.701$,所以基本可以认为理论上的分布与实验结果在置信度为70\%的条件下可信的。

\section{结论}
本实验结论是将光子视作粒子得到的能量随角度的分布与实验结果吻合,卡方检验结果与假设符合。微分散射截面随角度的分布与理论有一定差异,需要进一步探究原因。
\section{附录}
所有代码和处理结果可以从\url{https://github.com/greatofdream/AdvancePhysicsExperiment/tree/master/ComptonScatter}中获取。
\bibliography{report}
\end{document}