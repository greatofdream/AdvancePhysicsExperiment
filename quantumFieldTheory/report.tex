\documentclass[10pt]{ctexart}
\usepackage{graphicx, amsmath, amssymb, braket}
\usepackage{siunitx}
\usepackage{caption}
\usepackage{url}
\usepackage[version=4]{mhchem}
\title{群表示与量子场的运动方程}
\author{张爱强\\指导教师: }
\date{}
% set the section title format
% \ctexset{
%     section/format  += \raggedright
% }
% The following parameters seem to provide a reasonable page setup.
\topmargin 0.0cm
\oddsidemargin 0.2cm
\textwidth 16cm 
\textheight 21cm
\footskip 1.0cm
% set the abstract format
\newenvironment{sciabstract}{%
\begin{quote} \textbf{摘要: }}
{\end{quote}}
% set the bibliography format
\bibliographystyle{elsarticle-num}


\begin{document}
\maketitle
\begin{sciabstract}
    本文从Lorentz群表示出发,得到不同群表示对应的运动方程的差别。并从Weinberg的角度出发,验证不同群表示的运动方程。
    \par\textbf{关键词: } Lorentz群表示;量子场.
\end{sciabstract}

\section{引言}
量子场论中有三种对应基本粒子的场,包括标量量子场,矢量量子场,旋量量子场。标量量子场是最平庸的场,对应的基本粒子为Higgs粒子。
矢量量子场是各种非平庸量子场中有经典对应的最简单的场,对应于除引力外的其它相互作用力。旋量量子场是非平庸量子场中无经典对应的
最简单的场,已发现的所有物质型基本粒子都由旋量场描述。
\begin{figure}
    \centering
    \includegraphics[width=0.5\textwidth]{./fundenmentParticle.jpg}
    \caption{基本粒子}
    \label{fig:fundenmentParticle}
\end{figure}

量子场满足Klein-Gordon方程\cite{weinberg}:
\begin{equation}
    (\square-m^2)\Psi(x)=0
\end{equation}
此外矢量场满足
\begin{equation}
    \partial_\mu\psi^\mu(x)=0
\end{equation}
旋量场满足Dirac方程:
\begin{equation}
    (i\gamma^\mu\partial_\mu+m)\psi(x)=0
\end{equation}
本文试图从Lorentz群表示出发,给出不同场之间的运动方程的差异来源。并从量子场的函数形式出发,结合不同群表示的变换\cite{weinberg},验证上述结果。
\section{Lorentz群表示}
Lorentz群可以用矩阵$\Lambda^\mu_\nu$表示,构成GL(4,R)的子群O(3,1)\cite{yucheng}\cite{Zee}。
一般将Lorentz群限制为正规正时Lorentz群,$det\Lambda=1$,$\Lambda^0_0>1$,这些
子群构成SO(3,1).

通过无穷小变换\cite{weinberg},$\Lambda^\mu_\nu=\delta^\mu_\nu+\omega^\mu_\nu$,可以得到Poincare代数
\begin{align}
    i[J^{\mu\nu},J^{\rho\sigma}]&=\eta^{\nu\rho}J^{\mu\sigma}-\eta^{\mu\rho}J^{\nu\sigma}-\eta^{\sigma\mu}J^{\rho\nu}+\eta^{\sigma\nu}J^{\rho\mu}\\
    i[P^\mu,J^{\rho\sigma}]&=\eta^{\mu\rho}P^{\sigma}-\eta^{\mu\sigma}P^{\rho}
\end{align}

对于Lorentz群,Lie代数同构于so(3,1)代数:
\begin{equation}
    [J_i,J_j]=i\epsilon_{ijk}J_k,[J_i,K_j]=i\epsilon_{ijk}K_k,[K_i,K_j]=i\epsilon_{ijk}J_k
\end{equation}
其中$J_i$对应so(3)的代数,所以$so(3)\subset so(3,1)$\cite{yucheng}。so(3,1)没有非平庸的Abel理想,
所以so(3,1)是一个半单Lie代数。半单Lie代数总可以写成单李代数的直和,定义
\begin{equation}
    M_i=(J_i+iK_i)/2, N_i=(J_i-iK_i)/2
\end{equation}
对于上述两类变换满足
\begin{equation}
    [M_i,M_j]=i\epsilon_{ijk}M_k,[N_i,N_j]=i\epsilon_{ijk}N_k,[M_i,N_j]=0
\end{equation}
所以M,N分别张成so(3,1)的两个理想,而这两个理想对应的李代数与su(2)代数同构,(即旋转群的李代数o(3)同构),可以得到
\begin{equation}
    so(3,1)=su(2)\otimes su(2)
\end{equation}

因此我们可以通过使用SU(2)对应的不可约表示的直积来构成Lorentz群的表示。记SU(2)的不可约表示
$D^{(j)}$,对于Lorentz表示下的两个数为$(u,v)$,那么Lorentz的不可约表示为
\begin{equation}
    D^{(u,v)}=D^{(u)}\otimes D^{(v)}
\end{equation}
对应的基为$\ket{(u,v)}$,即SU(2)对应基的直积。
\begin{align}
    M_3 \ket{(u,v);m_u,m_v}=m_u\ket{(u,v);m_u,m_v}, N_3 \ket{(u,v)} =n_u \ket{(u,v);m_u,m_v}\\
    M_\pm \ket{(u,v);m_u,m_v} = \sqrt{u(u+1)-m_u(m_u\pm1)}\ket{(u,v);m_u\pm1,m_v}\\
    M_\pm \ket{(u,v);m_u,m_v} = \sqrt{u(u+1)-m_u(m_u\pm1)}\ket{(u,v);m_u,m_v\pm1}
\end{align}
由不同(u,v)得到的场总结如下\cite{Mceey}
\begin{table}
    \centering
    \begin{tabular}{|c|c|c|c|}
        $(j_+,j_-)$&维数&物理对应\\
        $(0,0)$&1&标量场\\
        $(\frac{1}{2},0)$&2&左手Weyl旋量场\\
        $(0,\frac{1}{2})$&2&右手Weyl旋量场\\
        $(\frac{1}{2},\frac{1}{2})$&4&4矢量场\\
        $(\frac{1}{2},0)\oplus (0,\frac{1}{2})$&4&Dirac旋量场\\
    \end{tabular}
\end{table}
\subsection{标量场}
标量场对应(u=0,v=0),表示为1维的,$D(\Lambda)=1$。标量场的基满足
\begin{equation}
    U(\Lambda)\psi(x)U^{-1}(\Lambda)=\psi(\Lambda x)
\end{equation}
\subsection{旋量场}
\subsubsection{Weyl旋量}
考虑左手Weyl旋量,表示空间基为$\ket{1/2,0},\ket{-1/2,0}$。在这组基下得到的生成元表示矩阵恰好为
Pauli矩阵。
\begin{equation}
    J_i=\frac{1}{2}\sigma_i, K_i=-\frac{i}{2}\sigma_i
\end{equation}

在Lorentz变化下的变换关系是
\begin{equation}
    D(\Lambda)\psi_{La}=e^{i\vec{\theta}\vec{J}+\vec{\phi}\vec{K}}\psi_{Lb}
\end{equation}

对于右手Weyl旋量,与左手恰好相反。

Weyl旋量在Lorentz变换下的关系是
\begin{align}
    U^{-1}(\Lambda)\psi_{La}(x)U(\Lambda)&=D_L(\Lambda)^{\ b}_{a}\psi_{Lb}(\Lambda^{-1}x)\\
    U^{-1}(\Lambda)\psi_{Ra}(x)U(\Lambda)&=D_R(\Lambda)^{\ b}_{a}\psi_{Rb}(\Lambda^{-1}x)
\end{align}
变换矩阵$D_L(\Lambda),D_R(\Lambda)$不是幺正的,旋量场算符非Hermitian,左右取厄密共轭:
\begin{equation}
    U^{-1}(\Lambda)\psi^\dagger_{La}(x)U(\Lambda)=D_R(\Lambda)^{\ b}_{a}\psi^\dagger_{Lb}(\Lambda^{-1}x)
\end{equation}
所以左手Weyl旋量的厄密共轭是右手Weyl旋量。

构造左手Weyl旋量场的Lagrangian。利用$\psi^\dagger_L,\psi_L$,要求Lagrangian是一个实标量,包含动力学和运动项,其中$\overline{\sigma}^{\mu}=(I,-\sigma)$
\begin{equation}
    L_L=i\psi^\dagger_L\overline{\sigma}^\mu\partial_\mu\psi_L-\frac{1}{2}m\psi_L\psi_L-\frac{1}{2}m\psi^\dagger_L\psi^\dagger_L
\end{equation}
对Lagrangian对应的作用量求变分,得到Weyl方程
\begin{equation}
    -i\overline{\sigma}^\mu\partial_\mu\psi_L+m\psi^\dagger_L=0
\end{equation}
同理可以得到右手Weyl方程:
\begin{equation}
    -i\overline{\sigma}^\mu\partial_\mu\psi_R+m\psi^\dagger_R=0
\end{equation}
\subsubsection{旋量场}
旋量场是$(\frac{1}{2},0)\oplus (0,\frac{1}{2})$的表示,那么对应的基底为
\begin{equation}
    \Psi=\left(\begin{array}{c}
        \psi_{La}\\
        \psi^{\dagger}_{LA}\\
    \end{array}\right)
\end{equation}
对应的运动方程为
\begin{equation}
    \left(\begin{array}{cc}
        m&-i\overline{\sigma}^\mu\partial_\mu\\
        -i\overline{\sigma}^\mu\partial_\mu&m\\
    \end{array}\right)\Psi=0
\end{equation}
定义$\gamma$矩阵为
\begin{equation}
    -i\left(\begin{array}{cc}
        0&\overline{\sigma}^\mu\\
        \overline{\sigma}^\mu&0\\
    \end{array}\right)
\end{equation}
那么即可得到旋量场对应的Dirac方程
\begin{equation}
    (\gamma^\mu\partial_\mu+m)\psi(x)=0
\end{equation}
\subsection{矢量场}
矢量场对应(u=1/2,v=1/2),表示为4维。可以看作$(\frac{1}{2},0)\otimes (0,\frac{1}{2})$的直积,对应基底为
\begin{equation}
    \left(\begin{array}{c}
        \psi^1_{L}\psi^1_R\\
        \psi^1_L\psi^2_R\\
        \psi^2_L\psi^1_R\\
        \psi^2_L\psi^2_R\\
    \end{array}\right)
\end{equation}
对应的表示可以写为两个SU(2)矩阵的直乘。

\begin{equation}
    U^{-1}(\Lambda)\psi(x)U(\Lambda)=\Lambda^{\mu}_{\ \nu} A^\nu
\end{equation}
矢量场的Lagranian为\cite{Lancaster}
\begin{equation}
    L=-\frac{1}{4}F_{\mu\nu}F^{\mu\nu}+\frac{1}{2}m^2A_\mu A^\mu
\end{equation}
对上式取散度可得到矢量场的运动方程为
\begin{equation}
    \partial_\mu\psi^\mu=0
\end{equation}
\section{量子场函数形式}
从粒子的场方程开始,根据Lorentz不变性导出函数内的系数,并在最后导出场方程。

利用产生湮灭场构建自由场,其中
\begin{align}
    \psi^+_l(x)=\sum_{\sigma n}{\int{d^3pu_l(x;\mathbf{p},\sigma,n)a(\mathbf{p},\sigma,n)}}\\
    \psi^-_l(x)=\sum_{\sigma n}{\int{d^3pv_l(x;\mathbf{p},\sigma,n)a^\dagger(\mathbf{p},\sigma,n)}}
\end{align}
根据$u_l,v_l$的不同值选取,可以得到不同的量子场\cite{weinberg}。其中标量场只有一个分量,满足Klein-Gordon方程\cite{weinberg}:
\begin{equation}
    (\square-m^2)\Psi(x)=0
\end{equation}
下面仅引用Weinberg关于矢量场和旋量场的结果。

\subsection{矢量场}
系数函数由零动量形式给定\cite{weinberg}
\begin{align}
    u^\mu(\mathbf{p},\sigma)=(\frac{m}{p^0})^{1/2}L(p)^\mu_{\ \nu}u^\nu(0,\sigma)\\
    v^\mu(\mathbf{p},\sigma)=(\frac{m}{p^0})^{1/2}L(p)^\mu_{\ \nu}v^\nu(0,\sigma)
\end{align}
自旋为1有质量时,$e^\mu(\vec{p},\sigma)=L^\mu_{\ \nu}(\vec{p})e^\mu(0,\sigma)$
\begin{equation}
    u^\mu(\vec{p},\sigma)=v^{\mu*}(\vec{p},\sigma)=(2p^0)^{-1/2}e^\mu(\vec{p},\sigma)
\end{equation}
用动量算符作用
\begin{equation}
    p_\mu e^\mu(\vec{p},\sigma)=0
\end{equation}
由上式可得
\begin{equation}
    \partial_\mu\psi^\mu=0
\end{equation}
\subsection{旋量场}
旋量场对应的变换满足
\begin{equation}
    D(L(p))\beta D^{-1}(L(p))=L^{\ 0}_\mu(p)\gamma^\mu=p_\mu\gamma^\mu/M
\end{equation}
那么可以推出
\begin{align}
    (p_\mu\gamma^\mu-M)u(\vec{p},\sigma)=0\\
    (p_\mu\gamma^\mu+M)v(\vec{p},\sigma)=0
\end{align}
从而结合旋量场的方程得到
\begin{equation}
    (i\gamma^\mu\partial_\mu-M)\psi(x)=0
\end{equation}
\section{结论}
矢量场和旋量场的运动方程不同,是因为它们具有不同的群表示,不同的群表示引起不可约张量算符不同的计算性质。
\bibliography{report}
\end{document}