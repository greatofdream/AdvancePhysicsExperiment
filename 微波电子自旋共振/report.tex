\documentclass[12pt]{ctexart}
\usepackage{graphicx, amsmath}
\usepackage[version=4]{mhchem}
\title{符合方法及原子核激发态寿命测量}
\author{张爱强\\指导教师: 王合英}

\date{}
% set the section title format
\ctexset{
    section/format  += \raggedright
}
% set the abstract format
\newenvironment{sciabstract}{%
\begin{quote} \textbf{摘要: }}
{\end{quote}}
% set the bibliography format
\bibliographystyle{elsarticle-num}


\begin{document}
\maketitle
\begin{sciabstract}
    本实验使用符合方法测量原子核激发态寿命。利用两个\ce{NaI}探测器对\ce{^{57}Co}放射源进行测量,用多道脉冲分析器
    测量不同能量范围的能谱。使用阈值甄别器筛选出\ce{^{57}Co}衰变后的\ce{^{56}Fe}激发态退激后
    产生的两条特征$\gamma$射线,使用符合器件对两条特征$\gamma$射线进行延时符合,拟合得到原子核激发态寿命。
    \par\textbf{关键词: } 阈值甄别; 符合方法.
\end{sciabstract}
\section{引言}
符合测量方法是物理实验从本底中选择信号的有效方法,在核物理实验中经常用来去除大量的低能本底。在Leins 和 Cowan的
反应堆测量中微子实验利用IBD反应$p+\overline{\nu_e}\to n+e^+$,对产物中中子俘获和电子湮灭的信号进行符合测量,捕捉到
极少的信号事例\cite{doi:10.1063/1.1770939}。选择时间上有关联的同时脉冲的符合为\texttt{瞬时符合},不同时但时间上
有延迟关系的符合为\texttt{延迟符合}。将实验上有关联的事件通过符合去掉是\texttt{反符合},比如在核物理实验中测量带电粒子能谱可以
利用康普顿散射的电子和光子进行反符合,以去除康普顿坪\cite{nuclear}。

探测器的种类有多种,包括气体探测器,闪烁体探测器等\cite{nuclear}。其中闪烁体探测器有液体闪烁体和固体闪烁体之分。\ce{NaI(Tl)}闪烁体探测器
是一种固体闪烁体,使用铊激活的碘化钠闪烁晶体。闪烁效率很高,而且易于制造,是常规测量$\gamma$射线能谱的2标准闪烁材料。但是\ce{NaI(Tl)}晶体的闪烁光
脉冲的主要成分的发光衰减事件常数是$230ns$,所以在高计数率和高事件分辨本领测量中的应用受限。

本实验利用闪烁体探测器和符合器件对\ce{^{56}Fe}激发态退激的两条特征$\gamma$射线进行符合,从而获得衰变事件的分布谱,对\ce{^{56}Fe}激发态的衰变时间
给出一个良好的估计。
\section{实验}
\subsection{实验仪器}
探测器为两个\ce{NaI(Tl)}闪烁体,探测器$\mathbf{I}$用厚窗和厚灵敏体积的闪烁体,可以过滤掉低能的$\gamma$信号;探测器$\mathbf{II}$用薄窗和薄体积的闪烁体,
可以检测到两种$\gamma$射线,但是对于高能的$\gamma$射线无法完全沉积能量,所以有能量的测量上限。

实验中多道采集探测器输出的能谱,阈值甄别器可以筛选不同大小的脉冲。符合器件可以对两路信号进行符合。TDC可以将时间差转换为幅度输出。
\subsection{实验内容}
\begin{enumerate}
    \item 对探测器$\mathbf{I}$和探测器$\mathbf{II}$的信号输出到多道,分别做相应的阈值选择,选择出对应的峰。
    \item 对TDC进行定标,之后用两路信号符合测量时间分布。
\end{enumerate}
\section{实验结果及讨论}
\subsection{两路信号能谱}

\section{结论}
本实验最终给出原子核衰变寿命为$\tau=8.229ns$。
\bibliography{report}
\end{document}